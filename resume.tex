% !TeX spellcheck = en_US
%% start of file `template.tex'.
%% Copyright 2006-2015 Xavier Danaux (xdanaux@gmail.com).
%
% This work may be distributed and/or modified under the
% conditions of the LaTeX Project Public License version 1.3c,
% available at http://www.latex-project.org/lppl/.


\documentclass[11pt,a4paper,sans]{moderncv}        % possible options include font size ('10pt', '11pt' and '12pt'), paper size ('a4paper', 'letterpaper', 'a5paper', 'legalpaper', 'executivepaper' and 'landscape') and font family ('sans' and 'roman')

% moderncv themes
\moderncvstyle{classic}                             % style options are 'casual' (default), 'classic', 'banking', 'oldstyle' and 'fancy'
\moderncvcolor{blue}                               % color options 'black', 'blue' (default), 'burgundy', 'green', 'grey', 'orange', 'purple' and 'red'
%\renewcommand{\familydefault}{\sfdefault}         % to set the default font; use '\sfdefault' for the default sans serif font, '\rmdefault' for the default roman one, or any tex font name
%\nopagenumbers{}                                  % uncomment to suppress automatic page numbering for CVs longer than one page

% character encoding
%\usepackage[utf8]{inputenc}                       % if you are not using xelatex ou lualatex, replace by the encoding you are using
%\usepackage{CJKutf8}                              % if you need to use CJK to typeset your resume in Chinese, Japanese or Korean

% adjust the page margins
\usepackage[scale=0.75]{geometry}
%\setlength{\hintscolumnwidth}{3cm}                % if you want to change the width of the column with the dates
%\setlength{\makecvtitlenamewidth}{10cm}           % for the 'classic' style, if you want to force the width allocated to your name and avoid line breaks. be careful though, the length is normally calculated to avoid any overlap with your personal info; use this at your own typographical risks...

% personal data
\name{Jason}{Huang}
\title{Resumé}                               % optional, remove / comment the line if not wanted
\address{4301 Romfield Cres}{L5M 4L4 Mississauga}{Canada}% optional, remove / comment the line if not wanted; the "postcode city" and "country" arguments can be omitted or provided empty
%\phone[mobile]{+1~(647)~631~6610}                   % optional, remove / comment the line if not wanted; the optional "type" of the phone can be "mobile" (default), "fixed" or "fax"
\phone[fixed]{+1~(416)~499~8887}
%\phone[fax]{+3~(456)~789~012}
\email{jasonhuang@disroot.org}                               % optional, remove / comment the line if not wanted
\homepage{jasonhuang.netlify.com}                         % optional, remove / comment the line if not wanted
%\social[linkedin]{jasonhuang03}                        % optional, remove / comment the line if not wanted
%\social[twitter]{jhthenerd}                             % optional, remove / comment the line if not wanted
\social[github]{jhthenerd}                              % optional, remove / comment the line if not wanted
%\social[gitlab]{jhthenerd}                              % optional, remove / comment the line if not wanted
%\extrainfo{Gitlab: jhthenerd}                 % optional, remove / comment the line if not wanted
%\photo[64pt][0.4pt]{picture}                       % optional, remove / comment the line if not wanted; '64pt' is the height the picture must be resized to, 0.4pt is the thickness of the frame around it (put it to 0pt for no frame) and 'picture' is the name of the picture file
\quote{Life is the ultimate optimization problem}                                 % optional, remove / comment the line if not wanted

% bibliography adjustements (only useful if you make citations in your resume, or print a list of publications using BibTeX)
%   to show numerical labels in the bibliography (default is to show no labels)
\makeatletter\renewcommand*{\bibliographyitemlabel}{\@biblabel{\arabic{enumiv}}}\makeatother
%   to redefine the bibliography heading string ("Publications")
%\renewcommand{\refname}{Articles}

% bibliography with mutiple entries
%\usepackage{multibib}
%\newcites{book,misc}{{Books},{Others}}
%----------------------------------------------------------------------------------
%            content
%----------------------------------------------------------------------------------
\begin{document}
%\begin{CJK*}{UTF8}{gbsn}                          % to typeset your resume in Chinese using CJK
%-----       resume       ---------------------------------------------------------
\makecvtitle

\section{Education}
\cventry{2017--Present}{High School}{John Fraser Secondary School}{Mississauga}{\textit{94\%}}{}  % arguments 3 to 6 can be left empty
%\cventry{year--year}{Degree}{Institution}{City}{\textit{Grade}}{Description}

%\section{Master thesis}
%\cvitem{title}{\emph{Title}}
%\cvitem{supervisors}{Supervisors}
%\cvitem{description}{Short thesis abstract}

\section{Experience}
%\subsection{Vocational}
\cventry{2017--2019}{Volunteer}{Central Library}{Mississauga}{}{Computer Buddies program member, general volunteer.% \newline{}%
%Achievements:%
	\begin{itemize}
		\item Communicate and teach older adults basic computer skills
		\item Work with clients to solve unique problems
		\item Maintain and help improve the experience for library patrons
	\end{itemize}
}
%\cventry{year--year}{Job title}{Employer}{City}{}{Description line 1\newline{}Description line 2}
%\subsection{Miscellaneous}
%\cventry{year--year}{Job title}{Employer}{City}{}{Description}

\section{Clubs}
\cventry{2018--Present}{Executive}{Computer Science Club}{John Fraser SS}{}{
	\begin{itemize}
		\item Taught other students advanced programming concepts
		\item Worked with other executives to promote computer science education
		\item Helped to plan and run a hackathon in 2 months with minimal preparation
	\end{itemize}
}
\cventry{2017--Present}{Co-founder}{3D Printing Club}{John Fraser SS}{}{
	\begin{itemize}
		\item Primary maintainer of 3D printers and related software
		\item Work within the school to promote and educate peers about 3D printing technology
		\item Collaborate with teachers to introduce 3D printing into the curriculum
	\end{itemize}
}
\cventry{2017--Present}{Lead Engineer}{VEX Robotics}{John Fraser SS}{}{
	\begin{itemize}
		\item Built, designed, and maintained robot hardware
		\item Collaborated with rest of team to resolve issues
		\item Worked with other high-ranking team members to recruit and vet new members
	\end{itemize}
}

\section{Honors \& Awards}
\cvitemwithcomment{2019}{Achievement in <Grade 10 Technological Design>}{John Fraser SS}
\cvitemwithcomment{2019}{Canadian Intermediate Math Contest (CIMC) School Champion}{John Fraser SS}
\cvitemwithcomment{2018}{Best Highschool Hack}{Redbull AdrenaLAN 2018}
%\cvitemwithcomment{2018}{Achievement in <Grade 9 Science - Advanced Placement>}{John Fraser SS}

%\newpage

\section{Scores}
%\cvitemwithcomment{2019}{SAT Physics (770)}{The College Board}
\cvitemwithcomment{2019}{AP Physics A (5)}{The College Board}
\cvitemwithcomment{2019}{AP Computer Science A (5)}{The College Board}
\cvitemwithcomment{2018}{SAT (1530)}{The College Board}

\section{Projects}
\cventry{2019--2019}{Dr. Dash}{}{Hack the North}{}{
	\url{https://github.com/davidli3100/HTN2019}
	\begin{itemize}
		\item A chatbot that helps patients submit pertinent information to doctors before their appointment, that interfaces with voice assistants, Facebook Messenger, and telephony
		\item Built using Firebase and DialogFlow, with a Python backend
	\end{itemize}
}
\cventry{2019--Present}{OpenDo}{}{Personal Project}{} {
	\url{https://gitlab.com/jhthenerd/XXSUNSET_opendo-go}
	\begin{itemize}
		\item An open-source GTD task manager built out of personal need
		\item Originally built in Electron and React, but pivoted to a CLI app written in Go
		\item Currently being ported to Rust
	\end{itemize}
}
\cventry{2019--On Hold}{PinBoard}{}{John Fraser SS}{}{
%	\url{}
	\begin{itemize}
		\item An interface allowing students to receive announcements, event updates, and polls from JFSS
		\item Developed using React Native and Firebase
	\end{itemize}
}
\cventry{2019--2020}{FraserHacks 2019 Website}{}{John Fraser SS}{}{
	\url{https://github.com/FraserHacks/FraserHacks-Website}
	\begin{itemize}
		\item A website promoting FraserHacks 2019
		\item Developed in pure HTML, CSS, JS
	\end{itemize}
}

\section{Skills}
\cvitem{Operating Systems}{Windows, Linux (Debian-based, RPM-based)}
\cvitem{Development Tools}{VS Code, Vim, Git}
\cvitem{Programming Languages}{C++, Java, Javascript, Go}
\cvitem{Soft Skills}{Organization, Leadership, Communication}
\cvitem{Strong Subjects}{Computer Science, Physics, Math}

\section{Interests}
\cvitem{CTFs}{Competitive infosec challenges where competitors use various red-team techniques to gain access to a flag (usually a specific string)}
%\cvitem{Photography}{I do casual photography for my family and occasionally post photos online}
\cvitem{Linux}{I use Linux on my daily driver laptop, and enjoy using and modifying the system in my free time}

%\section{Extra 1}
%\cvlistitem{Item 1}
%\cvlistitem{Item 2}
%\cvlistitem{Item 3. This item is particularly long and therefore normally spans over several lines. Did you notice the indentation when the line wraps?}
%
%\section{Extra 2}
%\cvlistdoubleitem{Item 1}{Item 4}
%\cvlistdoubleitem{Item 2}{Item 5\cite{book1}}
%\cvlistdoubleitem{Item 3}{Item 6. Like item 3 in the single column list before, this item is particularly long to wrap over several lines.}

\section{References}
References available upon request
%\begin{cvcolumns}
%  \cvcolumn{Category 1}{\begin{itemize}\item Person 1\item Person 2\item Person 3\end{itemize}}
%  \cvcolumn{Category 2}{Amongst others:\begin{itemize}\item Person 1, and\item Person 2\end{itemize}(more upon request)}
%  \cvcolumn[0.5]{All the rest \& some more}{\textit{That} person, and \textbf{those} also (all available upon request).}
%\end{cvcolumns}

% Publications from a BibTeX file without multibib
%  for numerical labels: \renewcommand{\bibliographyitemlabel}{\@biblabel{\arabic{enumiv}}}% CONSIDER MERGING WITH PREAMBLE PART
%  to redefine the heading string ("Publications"): \renewcommand{\refname}{Articles}
%\nocite{*}
%\bibliographystyle{plain}
%\bibliography{publications}                        % 'publications' is the name of a BibTeX file

% Publications from a BibTeX file using the multibib package
%\section{Publications}
%\nocitebook{book1,book2}
%\bibliographystylebook{plain}
%\bibliographybook{publications}                   % 'publications' is the name of a BibTeX file
%\nocitemisc{misc1,misc2,misc3}
%\bibliographystylemisc{plain}
%\bibliographymisc{publications}                   % 'publications' is the name of a BibTeX file

%\clearpage\end{CJK*}                              % if you are typesetting your resume in Chinese using CJK; the \clearpage is required for fancyhdr to work correctly with CJK, though it kills the page numbering by making \lastpage undefined
\end{document}


%% end of file `template.tex'.
